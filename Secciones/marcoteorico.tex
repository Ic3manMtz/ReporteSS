\section{Elementos de un modelo de movilidad}
Un modelo de movilidad grupal debe considerar los siguientes elementos:
\begin{itemize}
	\item \textbf{Agrupación}: Identificación de conjuntos de individuos que se desplazan de forma coordinada.
	\item \textbf{Trayectorias}: Patrones de movimiento a nivel individual y grupal.
	\item \textbf{Interacciones}: Dinámicas dentro del grupo y entre grupos.
	\item \textbf{Características estadísticas}: Velocidad, densidad, cohesión, dirección y separación entre individuos.
\end{itemize}

Recordando que una trayectoria individual tiene tres elementos básicos: \textit{puntos de recorrido}, \textit{tiempos de pausa} y \textit{longitud de vuelo}. Dentro del contexto de este proyecto las trayectorias individuales y grupales no cuentan con puntos de recorrido como tal, ya que el área de grabación es un paso común para comunicar diferentes zonas dentro de la universidad. Por lo que se considera que no hay puntos de recorrido; sin embargo, un punto que es interesante es saber si los grupos se detienen en algún punto del recorrido, y de ser así por cuanto tiempo lo hacen. Bajo esta premisa este comportamiento es el tiempo de pausa. \\
Otra consideración importante es que cuando un grupo cambia su densidad de individuos, se considera que el grupo original deja de existir y uno o varios grupos con la nueva densidad son creados.