La simulación de una red de comunicaciones en donde intervienen dispositivos personales de comunicación requiere contar con modelos que representen fielmente los patrones de movimiento de las personas. De otra manera, la utilidad de las conclusiones que se puedan obtener de esa simulación es limitada. Para avanzar hacia la definición de un modelo de movilidad humana grupal, se propone la construcción de una base de datos de videos aéreos —capturados por un dron— y su análisis mediante herramientas de IA, lo que permitirá determinar algunas características de la movilidad de interés.

En cuanto al proceso de diseño de un modelo de movilidad, es fundamental contar con una fuente de trazas que permita construir el modelo extrayendo las características necesarias.

El objetivo principal del proyecto es contar con una caracterización de los grupos humanos que se desplazan juntos, identificando las características estadísticas de los mismos. Entre los logros alcanzados se encuentran: la detección y el rastreo de individuos dentro de un video; la identificación de los patrones de movimiento y las interacciones entre individuos para la obtención de grupos; y la extracción de características relevantes sobre la movilidad y las interacciones grupales.