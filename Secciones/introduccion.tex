\section{Importancia de los modelos de movilidad para la evaluación de protocolos para redes móviles}
La simulaci\'on de una red de comunicaciones en donde intervienen dispositivos personales de comunucaci\'on requiere contar con modelos que representen fielmente los patrones de movimiento de las personas. De otra manera, la utilidad de las conclusiones que se puedan obtener de esa simulaci\'on es limitada. 

\noindent Para avanzar hacia la definici\'on de un modelo de movilidad humana grupal, se propone la construcci\'on de una base de datos de videos a\'ereos (capturados por un dron) y su an\'alisis v\'ia herramientas de IA. Esto nos permitir\'a determinar algunas caracter\'isticas de la movilidad de inter\'es.

\section{Proceso de diseño de un modelo de movilidad}
El diseño de un modelo de movilidad implica varias etapas:
\begin{itemize}
	\item \textbf{Recolección de datos reales}: Captura y organización de videos.
	\item \textbf{Procesamiento de videos}: Manipulación de los videos para la extracción de trayectorias individuales.
	\item \textbf{Identificación de grupos de individuos}: Análisis de las trayectorias individuales para la identificación de grupos de individuos.
	\item \textbf{Caracterización de los grupos de individuos identificados}: Análisis de los datos obtenidos de los grupos de individuos.
\end{itemize}

\section{Objetivo del proyecto}
El objetivo principal del proyecto es contar con una caracterización de los grupos humanos que se desplazan juntos. Identificando las características estadísticas de los mismos.

\section{Logros}
\begin{itemize}
	\item  Detectar y rastrear individuos dentro de un video.
	
	\item Encontrar los patrones de movimiento y las interacciones entre individuos para la obtención de grupos.
	
	\item  Extracción de características relevantes sobre la movilidad y las interacciones grupales
\end{itemize}
