\section{Importancia de los modelos de movilidad para la evaluación de protocolos para redes móviles}
La simulación de una red de comunicaciones en donde intervienen dispositivos personales de comunicación requiere de modelos que representen fielmente los patrones de movimiento de las personas. De lo contrario, las conclusiones derivadas de dicha simulación pueden ser poco útiles. Para avanzar hacia la definición de un modelo de trayectorias individuales, se propone caracterizar los datos de una base de datos existente que permita modelar trayectorias de una forma eficaz.

\section{Proceso de diseño de un modelo de movilidad}
El diseño de un modelo de movilidad implica varias etapas:
\begin{itemize}
	\item \textbf{Caracterización de datos}: Limpieza, depuración y análisis exploratorio.
	\item \textbf{Identificación de trayectorias}: Extracción de secuencias de movimiento significativas.
	\item \textbf{Validación y evaluación}: Medición de la calidad y representativividad de las trayectorias.
\end{itemize}

\section{Objetivo del proyecto}
El objetivo principal del proyecto es obtener una caracterización estadística de las trayectorias individuales a partir de un conjunto de datos de movilidad. Esto incluye la identificación de trayectorias peatonales y su análisis mediante herramientas de IA.

\section{Logros}
\begin{itemize}
	\item  Reducir el conjunto de datos de 69.98 millones de registros, 19 columnas y un peso de 22 GB a 51 millones de registros, 7 columnas y 7 GB.
	
	\item Identificar que el 68.73\% de los registros tienen precisión GPS satelital (1-20 metros).
	
	\item  Desarrollar un algoritmo de evaluación de calidad de trayectorias con métrica como volumen, cobertura temporal, precisión GPS y diversidad espacial.
	
	\item Encontrar un respaldo del algoritmo de evaluación usando el algoritmo de clusterización K Medias con los puntos de recorrido.
	
	\item Graficar la distribución de las longitudes de vuelo para poder aproximar una distribución probabilística.
\end{itemize}
