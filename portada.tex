%%%%%%   TIPO DE DOCUMENTO: Artículo   %%%%%%
\documentclass[letterpaper,11pt,spanish]{report}

% ========================================
% CONFIGURACIÓN BÁSICA DEL DOCUMENTO
% ========================================
\usepackage[utf8]{inputenc}        % Soporte para caracteres especiales y acentos
\usepackage[spanish]{babel}        % Idioma español (traducciones y reglas tipográficas)
\usepackage{xspace}                 % Manejo inteligente de espacios después de comandos
\renewcommand{\baselinestretch}{1} % Interlineado simple

% --------------------------
% CONFIGURACIÓN DE TÍTULOS 
% --------------------------

%\titleformat{comando}{forma}
%   {formato}
%   {etiqueta}
%   {separacion}
%   {antes del titulo}
%   [despues del titulo]

\usepackage{titlesec}
\usepackage{etoolbox}

\titleformat{\chapter}[display] 
{\Large\raggedleft} 
{}
{0pt}
{
	\ifnum\value{chapter}>0 
	\rule{\textwidth}{0.5pt}
	\vspace{1ex}
	{\Large\bfseries\chaptername\ \thechapter \\}
	\vspace{1ex}
	\fi
} 
[
\rule{\textwidth}{0.3pt}
] 


\titleformat{\section}
{\normalfont\bfseries\Large\raggedright}
{\thesection}
{1em}
{}

\titleformat{\subsection}
{\normalfont\bfseries\large\raggedright}
{\thesubsection}
{1em}
{}

% --------------------------
% CONFIGURACIÓN DE ENCABEZADOS CON FANCYHDR
% --------------------------
\usepackage{fancyhdr}

\renewcommand{\chaptermark}[1]{\markboth{#1}{}}
\fancypagestyle{mainstyle}{%
	\fancyhf{}
	\fancyhead[L]{\leftmark}
	\fancyhead[R]{\thepage}
	\fancyfoot{}
}

% --------------------------
% CONFIGURACIÓN DE ANEXOS
% --------------------------
\newcommand{\appendixtitleformat}{
	\titleformat{\chapter}[display]
	{\Large\raggedleft}
	{\appendixname\ \thechapter}
	{1ex}
	{\Large\bfseries}
	[\vspace{1ex}\rule{\textwidth}{0.3pt}]
}


% ========================================
% MATEMÁTICAS Y SÍMBOLOS
% ========================================
%\usepackage{amsmath}                % Entornos matemáticos avanzados
%\usepackage{amssymb}                % Símbolos matemáticos extendidos
%\usepackage{amscd}                  % Diagramas conmutativos
%\usepackage{amsthm}                 % Teoremas y entornos de demostración

% ========================================
% ALGORITMOS Y PSEUDOCÓDIGO
% ========================================
%\usepackage{algorithm}              % Entornos para algoritmos
%\usepackage{algpseudocode}          % Estilo para pseudocódigo

% ========================================
% GRÁFICOS E IMÁGENES
% ========================================
\usepackage[draft]{graphicx}               % Inclusión y manipulación de imágenes [draft]-> No inserta las imágenes en el PDF
\usepackage{subcaption}             % Subtítulos para subfiguras (subfigure)
%\usepackage{tikz}                   % Creación de gráficos vectoriales
%\usetikzlibrary{shapes, arrows}     % Formas y flechas para TikZ
%\usetikzlibrary{arrows.meta}        % Estilos avanzados de flechas
%\usetikzlibrary{positioning}        % Posicionamiento preciso de nodos

% ========================================
% CÓDIGO FUENTE Y LISTADOS
% ========================================
\usepackage{listings}               % Inclusión de código fuente
\renewcommand{\lstlistingname}{Código} % Cambia "Listing" por "Código"
\renewcommand{\lstlistlistingname}{Lista de Códigos} % Título de la lista de códigos
\lstset{                            % Configuración básica de listados:
	basicstyle=\ttfamily\small,     
	numbers=left,
	frame=shadowbox,
	breaklines=true,               
	captionpos=b,                   
}

% ========================================
% ESTRUCTURAS DE DIRECTORIOS
% ========================================
%\usepackage{dirtree}                % Generación de árboles de directorios

% ========================================
% LISTAS Y ENUMERACIONES
% ========================================
\usepackage{enumitem}               % Personalización de listas (espaciados, etiquetas)
\setlist{itemsep=0pt, topsep=5pt}   % Configuración de espacios en listas

% ========================================
% TABLAS Y ELEMENTOS FLOTANTES
% ========================================
%\usepackage{booktabs}               % Tablas con mejor formato
\usepackage{float}                  % Control avanzado de posición de flotantes
\usepackage{caption}                % Personalización de leyendas
%\usepackage{calc}                   % Cálculos precisos de dimensiones

% ========================================
% HIPERVÍNCULOS Y REFERENCIAS
% ========================================
\usepackage{hyperref}               % Hipervínculos en el documento
\usepackage[active]{srcltx}         % Búsqueda inversa en editores (doble clic en PDF)

% ========================================
% OTROS UTILITARIOS
% ========================================
\usepackage{verbatim}               % Entornos para texto sin interpretación
\usepackage{makeidx}                % Generación de índices

% ========================================
% CONFIGURACIONES ADICIONALES
% ========================================
\renewcommand{\thesection}{\thechapter.\arabic{section}} % Numeración arábiga para secciones
\renewcommand{\thesubsection}{\thechapter.\arabic{section}.\arabic{subsection}}

% --------------------------
% CONFIGURACIÓN DE PÁGINA
% --------------------------
\setlength{\textheight}{21.6cm}
\setlength{\textwidth}{14cm}
\setlength{\oddsidemargin}{1cm}
\setlength{\evensidemargin}{1cm}
\captionsetup[figure]{position=below, skip=0pt}

% --------------------------
% INICIO DEL DOCUMENTO
% --------------------------
\begin{document}
	
	% --------------------------
	% PORTADA
	% --------------------------
	\thispagestyle{empty}
	
	\begin{figure}[h]
		\centering
		\includegraphics[scale=0.6]{img/logoUAM.png}
	\end{figure}
	
	\begin{center}
		\rule{\textwidth}{0.5pt}
		\\[0.7em]
		{\LARGE\bfseries Caracterización de datos de trayectorias individuales}\\[1cm]
		{\normalsize\itshape Presentado por:}\\
		{\large Jorge Rafael Martínez Buenrostro}
		\rule{\textwidth}{0.5pt}
	\end{center}
	
	\vspace{2cm}
	\begin{center}
		{\large Asesora: Dra. Elizabeth Pérez Cortés}    
	\end{center}
	
	\vfill  
	\begin{center}
		México, CDMX, a \today
	\end{center}
	
	\vspace{2cm}
	
	\pagenumbering{roman}
	\setcounter{page}{0}
	\pagestyle{plain}
	% --------------------------
	% RESUMEN (ABSTRACT)
	% --------------------------
	%\chapter*{Resumen}
	%Una trayectoria se define como la secuencia de desplazamientos realizada por un individuo en movimiento, compuesta por tres elementos principales: \texttt{puntos de recorrido}, \texttt{tiempos de pausa} y \texttt{longitudes de vuelo}. Los puntos de recorrido corresponden a las ubicaciones o pasos específicos por los que transita el individuo. Los tiempos de pausa representan los intervalos durante los cuales el individuo permanece detenido en un mismo punto de recorrido. Por último, la longitud de vuelo se refiere a la distancia recorrida entre dos puntos de recorrido consecutivos. En el presente trabajo se describe el proceso de caracterización de datos de movilidad, es decir, el proceso de limpieza y depuración de la información, mediante el cual elimina aquellos campos y registros que no le aportan valor a la trayectoria individual. El objetivo es identificar la mayor cantidad de trayectorias individuales, para así poder crear un modelo que permita simular el movimiento de individuos.
	
	% --------------------------
	% ÍNDICE DE CONTENIDOS
	% --------------------------
	\newpage
	\renewcommand{\contentsname}{Contenido}
	\tableofcontents
	
	% --------------------------
	% LISTA DE FIGURAS
	% --------------------------
	\renewcommand{\listfigurename}{Lista de Figuras}
	\listoffigures
	
	% --------------------------
	% LISTA DE CODIGOS FUENTE
	% --------------------------
	\cleardoublepage
	\addcontentsline{toc}{chapter}{Lista de Códigos}
	\lstlistoflistings
	
	% --------------------------
	% CUERPO PRINCIPAL
	% --------------------------
	\cleardoublepage
	\pagestyle{mainstyle} % Aplica el estilo de encabezado definido
	\pagenumbering{arabic}
	\setcounter{page}{1} % Reinicia numeración después de índices
	
	% Capítulos 
	\chapter{Introducción del Proyecto}
	La simulación de una red de comunicaciones en donde intervienen dispositivos personales de comunicación requiere contar con modelos que representen fielmente los patrones de movimiento de las personas. De otra manera, la utilidad de las conclusiones que se puedan obtener de esa simulación es limitada. Para avanzar hacia la definición de un modelo de movilidad humana grupal, se propone la construcción de una base de datos de videos aéreos —capturados por un dron— y su análisis mediante herramientas de IA, lo que permitirá determinar algunas características de la movilidad de interés.

En cuanto al proceso de diseño de un modelo de movilidad, es fundamental contar con una fuente de trazas que permita construir el modelo extrayendo las características necesarias.

El objetivo principal del proyecto es contar con una caracterización de los grupos humanos que se desplazan juntos, identificando las características estadísticas de los mismos. Entre los logros alcanzados se encuentran: la detección y el rastreo de individuos dentro de un video; la identificación de los patrones de movimiento y las interacciones entre individuos para la obtención de grupos; y la extracción de características relevantes sobre la movilidad y las interacciones grupales.
	
	\chapter{Marco teórico}
	\section{Elementos de un modelo de movilidad}
Un modelo de movilidad grupal debe considerar los siguientes elementos:
\begin{itemize}
	\item \textbf{Agrupación}: Identificación de conjuntos de individuos que se desplazan de forma coordinada.
	\item \textbf{Trayectorias}: Patrones de movimiento a nivel individual y grupal.
	\item \textbf{Interacciones}: Dinámicas dentro del grupo y entre grupos.
	\item \textbf{Características estadísticas}: Velocidad, densidad, cohesión, dirección y separación entre individuos.
\end{itemize}

Recordando que una trayectoria individual tiene tres elementos básicos: \textit{puntos de recorrido}, \textit{tiempos de pausa} y \textit{longitud de vuelo}. Dentro del contexto de este proyecto las trayectorias individuales y grupales no cuentan con puntos de recorrido como tal, ya que el área de grabación es un paso común para comunicar diferentes zonas dentro de la universidad. Por lo que se considera que no hay puntos de recorrido; sin embargo, un punto que es interesante es saber si los grupos se detienen en algún punto del recorrido, y de ser así por cuanto tiempo lo hacen. Bajo esta premisa este comportamiento es el tiempo de pausa. \\
Otra consideración importante es que cuando un grupo cambia su densidad de individuos, se considera que el grupo original deja de existir y uno o varios grupos con la nueva densidad son creados.
		
	\chapter{Objetivos y metodología}
	\section{Objetivos}
\noindent El objetivo principal del proyecto es contar con una caracterización de los grupos de humanos que se desplazan juntos. Identificando las características estadísticas de los mismos. \\

\noindent Los objetivos particulares son:
\begin{itemize}
	\item Construir una base de datos de videos aéreos de grupos humanos.
	\item Usar un modelo de IA que permita identificar y caracterizar los grupos humanos en los videos.
	\item Analizar los patrones de movimiento y las interacciones entre los grupos humanos.
\end{itemize}

\section{Metodología}
El proceso se dividió en las siguientes etapas: \\

{\Large\bfseries 1. Captura y almacenamiento}: Grabación de videos con un dron y organizacion en base de datos.

\vspace{0.5cm}
{\Large\bfseries 2. Preprocesamiento}: Conversión de videos a frames y almacenamiento de metadatos.

\vspace{0.5cm}
{\Large\bfseries 3. Detección y seguimiento de individuos}: Uso de YOLO para identificar y rastrear personas en los frames.

\vspace{0.5cm}
{\Large\bfseries 4. Extracción de características}: Cálculo de atributos de movilidad grupal.





	
	\chapter{Desarrollo}
	El desarrollo de este proyecto se estructuró en cuatro fases interconectadas, optimizadas para el análisis eficiente y escalable de grandes volúmenes de datos de video. La principal modificación a la metodología inicial fue la transición de un procesamiento basado en frames a un procesamiento directo sobre el flujo de video (stream), eliminando cuellos de botella de E/S (Input/Output).

\section{Captura y almacenamiento}
Esta etapa se encarga de la grabación física de los videos aéreos y su almacenamiento inicial en el servidor. Los videos son capturados mediante un dron en áreas de tránsito común dentro del campus universitario, y posteriormente son transferidos a un directorio designado para su procesamiento. La organización inicial de los archivos es fundamental para facilitar el análisis automatizado posterior.

\section{Preprocesamiento y orquestación}
En la metodología inicial, esta fase incluía la conversión intensiva de videos a secuencias de imágenes (frames). Para maximizar la eficiencia y reducir la latencia, esta etapa fue refactorizada, reemplazando la conversión explícita por un procesamiento directo concurrente sobre el archivo de video.

\subsection{Implementación de la Concurrencia}
Para manejar múltiples archivos de video simultáneamente, se implementó un sistema basado en \texttt{concurrent.futures.ThreadPoolExecutor} en Python. Este orquestador es el punto de entrada que:

\begin{itemize}
	\item Escanea un directorio de entrada en busca de todos los archivos \texttt{.mp4}.
	\item Despacha cada video a un \textit{worker thread} dedicado para su análisis completo.
	\item Utiliza la potencia de la GPU (NVIDIA RTX A5000) de manera eficiente, limitando el número de workers concurrentes a 4, un número óptimo para balancear la carga de VRAM del modelo YOLOv11x y el rendimiento del sistema.
\end{itemize}

\section{Detección y seguimiento de individuos (Tracking Unificado)}
Esta etapa es la base del análisis, encargada de identificar y mantener la identidad de cada persona a lo largo de las secuencias de video.

\subsection{Arquitectura Híbrida YOLO + DeepSORT}
Para lograr la detección y el seguimiento con alta precisión, se utilizó una arquitectura de dos etapas:

\begin{itemize}
	\item \textbf{Detección de Objetos (YOLOv11x):}
	\begin{itemize}
		\item Se empleó la versión \textit{Extra Large} (\texttt{yolo11x.pt}) del modelo YOLOv11 (You Only Look Once). Este modelo fue seleccionado por su equilibrio entre alta precisión de localización (mAP) y velocidad de inferencia, esencial para el procesamiento en tiempo real.
		\item El modelo fue configurado para detectar exclusivamente la clase \texttt{persona} (clase 0), optimizando el consumo de recursos.
	\end{itemize}
	\item \textbf{Rastreo por Re-Identificación (DeepSORT):}
	\begin{itemize}
		\item Los \textit{bounding boxes} generados por YOLO son pasados al algoritmo DeepSORT (Deep Simple Online and Real-time Tracking). DeepSORT asigna un ID de seguimiento (\textit{Track ID}) persistente a cada individuo.
		\item El algoritmo se basa en el filtro de Kalman para predecir la posición futura y utiliza una red neuronal (como MobileNet, según la configuración) para extraer características de apariencia (\textit{features}) que permiten re-identificar a una persona que ha sido ocluida o ha salido brevemente del campo de visión.
	\end{itemize}
\end{itemize}

\subsection{Persistencia de Datos}
El resultado de este proceso (el \textit{Track ID} y las coordenadas $(x_1, y_1, x_2, y_2)$ para cada persona en cada frame) se guarda inmediatamente en la base de datos PostgreSQL en la tabla \texttt{FrameObjectDetection}.

\section{Extracción y caracterización de grupos}
Esta fase final emplea los datos de seguimiento guardados para identificar patrones de interacción social (grupos) y generar métricas estadísticas clave.

\subsection{Algoritmo de Agrupamiento PDF}
La lógica de agrupación se implementa a través de la clase \texttt{GroupTracker}, que replica el algoritmo de agrupamiento basado en:

\begin{itemize}
	\item \textbf{Proximidad Espacial:} Se considera que dos individuos están en proximidad si la distancia entre sus centroides (calculada en píxeles) es menor a un umbral predefinido (ej. 100 píxeles).
	\item \textbf{Persistencia Temporal ($\tau$):} Para que una interacción sea clasificada como un grupo, la proximidad debe mantenerse durante un número mínimo de frames (ej. 15 frames). Esto evita la detección de interacciones fugaces o cruces accidentales.
\end{itemize}

El algoritmo utiliza teoría de grafos, donde los individuos son nodos y los pares estables son aristas, para identificar los componentes conexos (grupos de 2 o más personas) en cada frame.

\subsection{Análisis de Métricas y Reporte Final}
Una vez que el seguimiento y el agrupamiento de un video han finalizado, el \textit{pipeline} procede a la generación del reporte. Esta fase consulta los datos recién almacenados en las tablas \texttt{GroupDetection} y \texttt{FrameObjectDetection} para calcular las siguientes estadísticas:

\begin{itemize}
	\item \textbf{Identificación:} Número total de individuos únicos rastreados.
	\item \textbf{Grupos Únicos:} Cantidad total de identificadores de grupo diferentes encontrados.
	\item \textbf{Tamaño Promedio:} Tamaño promedio de los grupos detectados.
	\item \textbf{Duración de Grupos:} Identificación de los grupos más persistentes (número de frames en los que el grupo fue detectado).
	\item \textbf{Métricas de Grupo:} Aunque no se muestran en el reporte simple, el modelo de datos está diseñado para almacenar métricas de movilidad grupal, como la dispersión (varianza de las distancias entre miembros) y la velocidad promedio del centroide del grupo.
\end{itemize}

Todos los hallazgos estadísticos y el resumen del proceso se guardan en un documento de texto plano (\texttt{.txt}) con la nomenclatura \texttt{REPORTE\_\{video\_name\}.txt}, cumpliendo con el requisito final del proyecto.
	
	\chapter{Resultados}
	
	\chapter{Conclusiones y Trabajo futuro}
	
	
	% Referencias
	\renewcommand{\bibname}{Referencias} % Título de la bibliografía
	\begin{thebibliography}{9}
		\bibitem{ref1} Autor referencia 1
		\bibitem{ref2} Autor referencia 2
		\bibitem{ref3} Autor referencia 3
	\end{thebibliography}
	
\end{document}